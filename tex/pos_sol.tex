\documentclass[12pt,a4paper]{article}
\usepackage[a4paper]{geometry}
\geometry{verbose,tmargin=1.5cm,bmargin=2cm,lmargin=2cm,rmargin=2cm}
\usepackage{amsmath,amsfonts,amssymb,amsthm,dsfont,mathtools}
\usepackage{cancel}
\usepackage{setspace}
\usepackage{booktabs}
\usepackage{hyperref}
\usepackage{graphicx}
\usepackage{bookmark}
\usepackage{verbatim}
\onehalfspacing
\usepackage[brazilian]{babel}
\usepackage{tikz, pgfplots}
\usepackage{tikz-3dplot}
\usepackage{tkz-fct, tkz-base, tkz-euclide} %tkz-fct chama gnuplot --> precisa compilar com a opção -shell-escape depois de pdflatex
\usepackage{animate}
\pgfplotsset{width=10cm,compat=1.17}
\usetikzlibrary{calc,angles,quotes,intersections}
\usetikzlibrary{decorations.pathmorphing}
\DeclareMathOperator{\sen}{sen}

\begin{document}
	{\bf Problema:} determinar a posição do sol como função do dia do ano, da hora local e da latitude.

	\begin{figure}[!ht]
		\centering
		\begin{tikzpicture}[transform shape, scale=1.3]
			\begin{axis}[view={-15}{-20}, % {angulo azimutal}{angulo zenital}
			  zmax=3,
			  xmax=3,
			  ymax=3,
			  % height=10cm,
			  xtick=\empty,
			  ytick=\empty,
			  ztick=\empty,
			  clip=false,
			  hide axis
			  ]
				\def\r{2}

				\draw[->,>= stealth, dashed] (0,-\r,0)--(0,\r+1,0) node[pos=1, left] {$Leste$};
				\draw[->,>= stealth, dashed] (-\r,0,0)--(\r+1,0,0) node[pos=1, right] {$Norte$};
				\draw[->,>= stealth, dashed] (0,0,0)--(0,0,\r+1) node[pos=1, above] {$Zênite$};

				\addplot3+[domain=pi/4:pi,samples=200,samples y=0,no marks, black, thick]
					({\r*cos(deg(x))}, {\r*sin(deg(x))}, {0});

				\addplot3+[domain=0:pi,samples=200,samples y=0,no marks, black, thick]
					({\r*cos(deg(x))}, {0}, {\r*sin(deg(x))});

				\addplot3+[domain=0:pi/2,samples=200,samples y=0,no marks, black, thick]
					({0}, {\r*cos(deg(x))}, {\r*sin(deg(x))});

				\addplot3+[domain=0:pi/4,samples=200,samples y=0, no marks, blue, thick]
					({\r*cos(deg(x))}, {\r*sin(deg(x))}, {0});

				\addplot3+[domain=0:pi/4,samples=200,samples y=0, no marks, red, thick]
					({\r*sin(deg(x))/sqrt(2)}, {\r*sin(deg(x))/sqrt(2)}, {\r*cos(deg(x))});

				\addplot3+[domain=pi:2*pi,samples=200,samples y=0,no marks, black, dashed]
					({\r*cos(deg(x))}, {\r*sin(deg(x))}, {0});

				\addplot3+[domain=pi/2:pi,samples=200,samples y=0,no marks, black, dashed]
					({0}, {\r*cos(deg(x))}, {\r*sin(deg(x))});

				\addplot3+[domain=pi/4:pi/2,samples=200,samples y=0, no marks, black, dashed]
					({\r*sin(deg(x))/sqrt(2)}, {\r*sin(deg(x))/sqrt(2)}, {\r*cos(deg(x))});

				\FPeval\R{clip(\r/root(2,2))}
				\draw[thick,->,>= stealth] (0,0,0)--(\r/2,\r/2,\R) node[pos=1, right] {$\hat{s}$};
				\FPeval\xyprojection{\r/root(2,2)}
				\draw[dashed] (0,0,0)--(\xyprojection,\xyprojection,0);
				\path (\xyprojection+0.4,\xyprojection,0.1) node[below right, blue] {$A$};
				\path (\r/2,\r/2,\R+0.4) node[above, red] {$z$};

			\end{axis}
		\end{tikzpicture}
		\caption{Sistema de coordenadas horizontal}%
	\end{figure}

	Queremos descrever o vetor $\hat{s}$ ao longo do ano (NDA), ao longo da
	hora local e também sua dependência com a latitude ($\phi$). Para isso
	precisamos expressar como os ângulos $\color{red}{z}$ ({\bf distância zenital}) e
	$\color{blue}{A}$ ({\bf azimute}) dependem dessas variáveis.
	
\end{document}
