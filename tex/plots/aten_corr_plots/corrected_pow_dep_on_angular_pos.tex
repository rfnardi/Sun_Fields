
\documentclass[a4paper,12pt]{article}

\usepackage[T1]{fontenc}
\usepackage[utf8]{inputenc}
\usepackage[a4paper]{geometry}
\geometry{verbose,tmargin=1.5cm,bmargin=2cm,lmargin=2cm,rmargin=2cm}
\usepackage{amsmath,amsfonts,amssymb,amsthm,dsfont,mathtools}
\usepackage[brazilian]{babel} 

\usepackage{tikz, pgfplots} 
\usepackage{tkz-fct, tkz-base, tkz-euclide} 
\pgfplotsset{width=10cm,compat=1.18}


\begin{document}

\bigskip
\bigskip
\bigskip

\begin{figure}[!htb]
	\centering 
	\begin{tikzpicture}
		\begin{axis}[
			height=16cm, 
			width=16cm, 
			grid=both, 
			grid style={line width=.1pt, draw=gray!20},
			major grid style={line width=.2pt,draw=gray!50}, 
			minor tick num=4, 
			xlabel={$\theta_{Azim}$}, 
			ylabel={Potência refletida}, 
			% title={Intensidade da radiação de comprimento de onda $\lambda$}
			]
			\addplot[mark size=0.5pt, color=blue] table[x=thetaDeg, y=power, col sep=semicolon] 
				{../../../data/aten_corr_data/corrected_pow_dep_on_angular_pos.dat};
		\end{axis}
	\end{tikzpicture}
	\caption{Dependência da potência refletida com o ângulo azimutal. Valores considerados ao meio dia.} 
\end{figure}
\bigskip


\begin{figure}[!htb]
	\centering 
	\begin{tikzpicture}
		\begin{axis}[
			height=16cm, 
			width=16cm, 
			grid=both, 
			grid style={line width=.1pt, draw=gray!20},
			major grid style={line width=.2pt,draw=gray!50}, 
			minor tick num=4, 
			xlabel={$\theta_{Azim}$}, 
			ylabel={Percentual refletido}, 
			% title={Intensidade da radiação de comprimento de onda $\lambda$}
			]
			\addplot[mark size=0.5pt, color=blue] table[x=thetaDeg, y=ReflecPercent, col sep=semicolon] 
				{../../../data/aten_corr_data/corrected_pow_dep_on_angular_pos.dat};
		\end{axis}
	\end{tikzpicture}
	\caption{Dependência da potência refletida com o ângulo azimutal.
	Percentual em relação ao valor da radiação que chega ao espelho.} 
\end{figure}
\bigskip


% thetaDeg;power;ReflecPercent
% -90;784.168;70.7107
% -89.427;786.254;70.8988


\end{document}
