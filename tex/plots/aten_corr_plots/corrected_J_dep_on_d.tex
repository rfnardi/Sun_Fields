
\documentclass[a4paper,12pt]{article}

\usepackage[T1]{fontenc}
\usepackage[utf8]{inputenc}
\usepackage[a4paper]{geometry}
\geometry{verbose,tmargin=1.5cm,bmargin=2cm,lmargin=2cm,rmargin=2cm}
\usepackage{amsmath,amsfonts,amssymb,amsthm,dsfont,mathtools}
\usepackage[brazilian]{babel} 

\usepackage{tikz, pgfplots} 
\usepackage{tkz-fct, tkz-base, tkz-euclide} 
\pgfplotsset{width=10cm,compat=1.18}


\begin{document}

\bigskip
\bigskip
\bigskip

\begin{figure}[!htb]
	\centering 
	\begin{tikzpicture}
		\begin{axis}[
			height=16cm, 
			width=16cm, 
			grid=both, 
			grid style={line width=.1pt, draw=gray!20},
			major grid style={line width=.2pt,draw=gray!50}, 
			minor tick num=4, 
			xlabel={número de partículas em suspensão por $cm^3$}, 
			ylabel={J}, 
			title={Atenuação da radiação direta por espalhamento por partículas de aerosol}
			]
			\addplot[mark size=0.5pt, color=blue] table[x=d, y=J, col sep=semicolon] 
				{../../../data/aten_corr_data/corrected_J_dep_on_d.dat};
		\end{axis}
	\end{tikzpicture}
	\caption{Variação da intensidade da radiação em função da presença de partículas em suspensão.} 
\end{figure}

\bigskip


\begin{figure}[!htb]
	\centering 
	\begin{tikzpicture}
		\begin{axis}[
			height=16cm, 
			width=16cm, 
			grid=both, 
			grid style={line width=.1pt, draw=gray!20},
			major grid style={line width=.2pt,draw=gray!50}, 
			minor tick num=4, 
			xlabel={número de partículas em suspensão por $cm^3$}, 
			ylabel={Percentual}, 
			title={Atenuação da radiação direta por espalhamento por partículas de aerosol}
			]
			\addplot[mark size=0.5pt, color=blue] table[x=d, y=ReflPercent, col sep=semicolon] 
				{../../../data/aten_corr_data/corrected_J_dep_on_d.dat};
		\end{axis}
	\end{tikzpicture}
	\caption{Variação percentual da intensidade da radiação comparada ao caso de ar puro.} 
\end{figure}

% d;J;ReflPercent
% 0;889.346;100
% 1;889.205;99.9842

\end{document}
