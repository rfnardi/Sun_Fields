\documentclass[a4paper,12pt]{article}

\usepackage[T1]{fontenc}
\usepackage[utf8]{inputenc}
\usepackage[a4paper]{geometry}
\geometry{verbose,tmargin=1.5cm,bmargin=2cm,lmargin=2cm,rmargin=2cm}
\usepackage{amsmath,amsfonts,amssymb,amsthm,dsfont,mathtools}
\usepackage[brazilian]{babel} 

\usepackage{tikz, pgfplots} 
\usepackage{tkz-fct, tkz-base, tkz-euclide} 
\pgfplotsset{width=10cm,compat=1.17}


\begin{document}

\bigskip
\bigskip
\bigskip

\begin{figure}[!htb]
	\centering 
	\begin{tikzpicture}
		\begin{axis}[
			height=16cm, 
			width=16cm, 
			grid=both, 
			grid style={line width=.1pt, draw=gray!20},
			major grid style={line width=.2pt,draw=gray!50}, 
			minor tick num=4, 
			xlabel={x}, 
			ylabel={Power ($W$)}, 
			title={Potência refletida variando em termos da posição angular}]
			\addplot[mark size=0.5pt, color=blue] table[x=theta, y=power, col sep=semicolon] {../../../data/bare_data/pow_dep_mirror_angular_pos.dat};
		\end{axis}
	\end{tikzpicture}
	\caption{Potência refletida por um espelho a $10m$ de distância do ponto focal variando com a posição angular theta} 
\end{figure}


\end{document}
