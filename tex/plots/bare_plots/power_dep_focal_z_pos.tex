\documentclass[a4paper,12pt]{article}

\usepackage[T1]{fontenc}
\usepackage[utf8]{inputenc}
\usepackage[a4paper]{geometry}
\geometry{verbose,tmargin=1.5cm,bmargin=2cm,lmargin=2cm,rmargin=2cm}
\usepackage{amsmath,amsfonts,amssymb,amsthm,dsfont,mathtools}
\usepackage[brazilian]{babel} 

\usepackage{tikz, pgfplots} 
\usepackage{tkz-fct, tkz-base, tkz-euclide} 
\pgfplotsset{width=10cm,compat=1.17}


\begin{document}

\bigskip
\bigskip
\bigskip

\begin{figure}[!htb]
	\centering 
	\begin{tikzpicture}
		\begin{axis}[
			height=16cm, 
			width=16cm, 
			grid=both, 
			grid style={line width=.1pt, draw=gray!20},
			major grid style={line width=.2pt,draw=gray!50}, 
			minor tick num=4, 
			xlabel={x}, 
			ylabel={Power ($W$)}, 
			title={Potência refletida variando em termos da altura do ponto focal}]
			\addplot[mark size=0.5pt, color=red] table[x=z_pos, y=power, col sep=semicolon] {../../../data/bare_data/pow_dep_focal_z_pos.dat};
		\end{axis}
	\end{tikzpicture}
	\caption{Potência refletida por um espelho $10m$ ao sul variando com a altura do ponto focal em relação ao nível dos espelhos} 
\end{figure}

Observa-se que um pico de potência é atingido por volta da altura $16m$.
Considerando que o espelho está $10m$ ao sul da torre que sustenta o ponto
focal, é possível calcular o ângulo $\alpha$ formado pela vertical com o vetor que liga
o espelho ao ponto focal. Este ângulo deve coincidir com o próprio ângulo
zenital $z$ para esta posição de altura $16m$ do ponto focal:

\begin{equation}
	\tan(\alpha) = \frac{10}{16} = 0,625 \implies \alpha = 0.558
\end{equation}

O resultado bate com a conta do algoritmo que dá um valor de $ zen = 0.561$.
Claro que um resultado melhor pode ser obtido se determinarmos exatamente qual
é a altura onde ocorre a potência máxima (aqui pegamos $16m$ de olho), mas já indica 
que as confirmações simples que esperamos da trigonometria estão acontecendo.

\end{document}
