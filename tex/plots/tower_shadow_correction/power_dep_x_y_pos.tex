\documentclass[a4paper,12pt]{article}

\usepackage[T1]{fontenc}
\usepackage[utf8]{inputenc}
\usepackage[a4paper]{geometry}
\geometry{verbose,tmargin=1.5cm,bmargin=2cm,lmargin=2cm,rmargin=2cm}
\usepackage{amsmath,amsfonts,amssymb,amsthm,dsfont,mathtools}
\usepackage[brazilian]{babel} 

\usepackage{tikz, pgfplots} 
\usepackage{tkz-fct, tkz-base, tkz-euclide} 
\pgfplotsset{width=10cm,compat=1.17}


\begin{document}

\bigskip
\bigskip
\bigskip

\begin{figure}[!htb]
	\centering 
	\begin{tikzpicture}
		\begin{axis}[
			grid=both, 
			xlabel={x}, 
			ylabel={y}, 
			zlabel={Power}
			title={Potência refletida em termos da posição (x,y)}]
			\addplot3[surf,mesh/rows=100] table[x=x, y=y, z=power, col sep=colon] {../../../data/tower_shadow_correction/square_grid.dat};
		\end{axis}
	\end{tikzpicture}
	\caption{Contribuição de uma malha quadrada de heliostatos ao meio dia.} 
\end{figure}


\end{document}
