\documentclass[a4paper,12pt]{article}

\usepackage[T1]{fontenc}
\usepackage[utf8]{inputenc}
\usepackage[a4paper]{geometry}
\geometry{verbose,tmargin=1.5cm,bmargin=2cm,lmargin=2cm,rmargin=2cm}
\usepackage{amsmath,amsfonts,amssymb,amsthm,dsfont,mathtools}
\usepackage{cancel}
\usepackage{setspace}
\usepackage{hyperref} %para inserir links internos e para a internet
\usepackage{graphicx} %para inserir figuras
\onehalfspacing
\usepackage[brazilian]{babel} 

\usepackage{tikz, pgfplots} 
\usepackage{tkz-fct, tkz-base, tkz-euclide} 
\pgfplotsset{width=10cm,compat=1.17}
\usetikzlibrary{calc,angles,quotes,intersections}
\usetikzlibrary{decorations.pathmorphing}
\usetikzlibrary{3d}

\usepackage{fancyvrb} 



\begin{document}

\bigskip
\bigskip
\bigskip

\begin{figure}[!htb]
	\centering %para centralizar a figura na página
	\begin{tikzpicture}
		\begin{axis}[
			height=16cm, 
			width=16cm, 
			grid=both, 
			grid style={line width=.1pt, draw=gray!20},
			major grid style={line width=.2pt,draw=gray!50}, 
			minor tick num=4, 
			xlabel={NDA}, 
			ylabel={Power ($W$)}, 
			title={Potência refletida ao longo do ano}]
			
			\addplot[mark size=0.5pt, color=blue] table[x=NDA, y=POWER, col sep=semicolon] {../src/pot_dep_NDA.dat};

		\end{axis}
	\end{tikzpicture}
	\caption{Dependência da potência refletida por um espelho com o NDA} 
\end{figure}



\end{document}
