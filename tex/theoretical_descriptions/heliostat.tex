\documentclass[12pt,a4paper]{article}
\usepackage[a4paper]{geometry}
\geometry{verbose,tmargin=1.5cm,bmargin=2cm,lmargin=2cm,rmargin=2cm}
\usepackage{amsmath,amsfonts,amssymb,amsthm,dsfont,mathtools}
\usepackage{cancel}
\usepackage{setspace}
\usepackage{booktabs}
\usepackage{hyperref}
\usepackage{graphicx}
\usepackage{bookmark}
\usepackage{verbatim}
\onehalfspacing
\usepackage[brazilian]{babel}
\usepackage{tikz, pgfplots}
\usepackage{tikz-3dplot}
\usepackage{tkz-fct, tkz-base, tkz-euclide} %tkz-fct chama gnuplot --> precisa compilar com a opção -shell-escape depois de pdflatex
\usepackage{animate}
\pgfplotsset{width=10cm,compat=1.18}
\usetikzlibrary{calc,angles,quotes,intersections}
\usetikzlibrary{decorations.pathmorphing}
\DeclareMathOperator{\sen}{sen}

\begin{document}

\begin{center} {\bf \Large Movimentos do heliostato} \end{center}

\bigskip
\bigskip
\bigskip
\bigskip
\bigskip
\bigskip
\bigskip

	\begin{figure}[!htb]
		\centering
		\begin{animateinline}[poster=first,controls={play,stop,step}]{8}
			\multiframe{50}{ri=0+2} {
			\begin{tikzpicture}[scale=1.3]
				% \useasboundingbox (-1.6, 0) rectangle (8, 8);
				\begin{axis}[
					view={-20}{-20}, % {angulo azimutal}{angulo zenital}
					zmax=3, 
					xmax=3, 
					ymax=3,
					% height=10cm,
					xtick=\empty, 
					ytick=\empty, 
					ztick=\empty,
					clip=false, 
					hide axis 
					]
					\def\r{1}

					% DISCO (ENGRENAGEM) NO EIXO VERTICAL
					\addplot3+[domain=pi:2*pi,samples=200,samples y=0,no marks, black, fill=gray!50]
						({\r*cos(deg(x))}, {\r*sin(deg(x))}, {0});

					%EIXO VERTICAL
					\def\altEixo{2}
					\draw[very thick] (0,0,-0.8)--(0,0,\altEixo) ;

					% DISCO (ENGRENAGEM) NO EIXO VERTICAL
					\addplot3+[domain=0:pi,samples=200,samples y=0,no marks, black, fill=gray!50]
						({\r*cos(deg(x))}, {\r*sin(deg(x))}, {0});

					\draw[->, >=stealth, dashed, very thin] (0,0,\altEixo)--(0,0,\altEixo + 1.5) ;

					%"GARFO"
					\def\altTotal{3.5}
					\FPeval\angPhi{(\ri)*pi/180}
					\FPeval\senPhi{sin(\angPhi)}
					\FPeval\cosPhi{cos(\angPhi)}
					\draw[very thick] (-\cosPhi,\senPhi,\altEixo)--(\cosPhi,-\senPhi,\altEixo) ;
					\draw[very thick] (-\cosPhi,\senPhi,\altEixo)--(-\cosPhi,\senPhi,\altTotal) ;
					\draw[very thick] (\cosPhi,-\senPhi,\altEixo)--(\cosPhi,-\senPhi,\altTotal) ;

					% EIXO DE ROTAÇÃO HORIZONTAL
					\draw[thick] (-\cosPhi,\senPhi,\altTotal)--(\cosPhi,-\senPhi,\altTotal) ;

					\FPeval\angTheta{0+20*pi/180}
					\FPeval\senTheta{sin(\angTheta)}
					\FPeval\cosTheta{cos(\angTheta)}


					%					(y   , x  , z    )
					\coordinate (A) at (-0.95*\cosPhi +\senPhi*1.5*\cosTheta,1.5*\cosTheta*\cosPhi +0.95*\senPhi,\altTotal - 1.5*\senTheta);
					\coordinate (B) at (-0.95*\cosPhi -\senPhi*1.5*\cosTheta,-1.5*\cosTheta*\cosPhi +0.95*\senPhi,\altTotal + 1.5*\senTheta);
					\coordinate (C) at (0.95*\cosPhi -\senPhi*1.5*\cosTheta,-1.5*\cosTheta*\cosPhi - 0.95*\senPhi,\altTotal + 1.5*\senTheta);
					\coordinate (D) at (0.95*\cosPhi +\senPhi*1.5*\cosTheta,1.5*\cosTheta*\cosPhi - 0.95*\senPhi,\altTotal - 1.5*\senTheta);

					\draw[thin, fill=blue!10] (A)--(B)--(C)--(D)--cycle;

					% EIXOS HORIZONTAIS
					\draw[->, >=stealth, dashed, very thin, color=red] (-2,0,0)--(2,0,0) ;
					\draw[->, >=stealth, dashed, very thin, color=blue] (0,-2,0)--(0,2,0) ;

					% EIXO VERTICAL (PARTE ACIMA DO ESPELHO)
					\draw[->, >=stealth, dashed, very thin] (0,0,\altEixo+1.5)--(0,0,5) ;

				\end{axis}
			\end{tikzpicture} }
		\end{animateinline}
		\caption{Heliostato --- Movimento no eixo vertical}%
	\end{figure}

	\newpage

\begin{center} {\bf \Large Movimentos do heliostato} \end{center}

\bigskip
\bigskip
\bigskip
\bigskip
\bigskip
\bigskip
\bigskip

	\begin{figure}[!htb]
		\centering
		\begin{animateinline}[poster=first,controls={play,stop,step}]{8}
			\multiframe{20}{ri=0+2}
			{
				\begin{tikzpicture}[transform shape, scale=1.3]
				% \useasboundingbox (-8, -8) rectangle (8, 8); %not working
					\begin{axis}[view={-20}{-20}, % {angulo azimutal}{angulo zenital}
						zmax=3,
						xmax=3,
						ymax=3,
						% height=10cm,
						xtick=\empty,
						ytick=\empty,
						ztick=\empty,
						clip=false,
						hide axis
						]
						\def\r{1}

						% DISCO (ENGRENAGEM) NO EIXO VERTICAL
						\addplot3+[domain=pi:2*pi,samples=200,samples y=0,no marks, black, fill=gray!50]
							({\r*cos(deg(x))}, {\r*sin(deg(x))}, {0});

						%LIMITES DO DESENHO 
						\draw[thin, dashed] (0,0,0)--(0,0,5);
						\draw[thin, dashed] (-2,0,0)--(2,0,0);

						%EIXO VERTICAL
						\def\altEixo{2}
						\draw[very thick] (0,0,-0.8)--(0,0,\altEixo);

						% DISCO (ENGRENAGEM) NO EIXO VERTICAL
						\addplot3+[domain=0:pi,samples=200,samples y=0,no marks, black, fill=gray!50]
							({\r*cos(deg(x))}, {\r*sin(deg(x))}, {0});

						%"GARFO"
						\def\altTotal{3.5}
						\draw[very thick] (-1,0,\altEixo)--(1,0,\altEixo) ;
						\draw[very thick] (-1,0,\altEixo)--(-1,0,\altTotal);
						\draw[very thick] (1,0,\altEixo)--(1,0,\altTotal);

						% EIXO VERTICAL SOB O ESPELHO
						\draw[->, >=stealth, dashed, very thin] (0,0,\altEixo)--(0,0,\altEixo + 1.5) ;

						% EIXO DE ROTAÇÃO HORIZONTAL
						\draw[thick] (-1,0,\altTotal)--(1,0,\altTotal) ;

						\FPeval\angTheta{0+\ri*pi/180}
						\FPeval\senTheta{sin(\angTheta)}
						\FPeval\cosTheta{cos(\angTheta)}

						%					(y   , x  , z    )
						\coordinate (A) at (-0.95 ,1.5*\cosTheta,\altTotal - 1.5*\senTheta);
						\coordinate (B) at (-0.95 ,-1.5*\cosTheta,\altTotal + 1.5*\senTheta);
						\coordinate (C) at (0.95 ,-1.5*\cosTheta,\altTotal + 1.5*\senTheta);
						\coordinate (D) at (0.95 ,1.5*\cosTheta,\altTotal - 1.5*\senTheta);

						\draw[thin, fill=blue!10] (A)--(B)--(C)--(D)--cycle;

						% DISCO (ENGRENAGEM) NO EIXO HORIZONTAL
						\def\raioSup{0.6}
						\addplot3+[domain=0:2*pi,samples=200,samples y=0,no marks, black, fill=gray!50]
							({1.2}, {\raioSup*sin(deg(x))}, {\raioSup*cos(deg(x)) + \altTotal} );

						% EIXO DE ROTAÇÃO HORIZONTAL (PARTE VISÍVEL)
						\draw[thick] (1.2,0,\altTotal)--(1.3,0,\altTotal) ;

						% EIXOS HORIZONTAIS
						\draw[->, >=stealth, dashed, very thin, color=red] (-2,0,0)--(2,0,0) ;
						\draw[->, >=stealth, dashed, very thin, color=blue] (0,-2,0)--(0,2,0) ;

						%EIXO VERTICAL SOBRE O ESPELHO
						\draw[->, >=stealth, dashed, very thin] (0,0,\altEixo+1.5)--(0,0,5) ;

					\end{axis}
				\end{tikzpicture}
			}
		\end{animateinline}
		\caption{Heliostato --- movimento do eixo horizontal}%
		\end{figure}

		\newpage

\begin{center} {\bf \Large Movimentos do heliostato} \end{center}

\bigskip
\bigskip
\bigskip
\bigskip
\bigskip
\bigskip
\bigskip

		\begin{figure}[!htb]
			\centering
			\begin{tikzpicture}[transform shape, scale=1.3]
				\begin{axis}[view={-20}{-20}, % {angulo azimutal}{angulo zenital}
					zmax=3,
					xmax=3,
					ymax=3,
					% height=10cm,
					xtick=\empty,
					ytick=\empty,
					ztick=\empty,
					clip=false,
					hide axis
					]
					\def\r{1}

					%BASE 1
					\coordinate (BASE1_sub_a) at (-1.5,-1.5,-0.7);
					\coordinate (BASE1_sub_b) at (1.5,-1.5,-0.7);
					\coordinate (BASE1_sub_c) at (1.5,1.5,-0.7);
					\coordinate (BASE1_sub_d) at (-1.5,1.5,-0.7);
					\coordinate (BASE1_a) at (-1.5,-1.5,-0.6);
					\coordinate (BASE1_b) at (1.5,-1.5,-0.6);
					\coordinate (BASE1_c) at (1.5,1.5,-0.6);
					\coordinate (BASE1_d) at (-1.5,1.5,-0.6);
					\coordinate (BASE1_inner_a) at (-1.4,-1.4,-0.7);
					\coordinate (BASE1_inner_b) at (1.4,-1.4,-0.7);
					\coordinate (BASE1_inner_c) at (1.4,1.4,-0.7);
					\coordinate (BASE1_inner_d) at (-1.4,1.4,-0.7);

					\tkzDrawSegment(BASE1_a,BASE1_b)
					\tkzDrawSegment(BASE1_a,BASE1_d)
					\tkzDrawSegment(BASE1_inner_a,BASE1_inner_b)
					\tkzDrawSegment(BASE1_inner_a,BASE1_inner_d)
					\tkzDrawSegment(BASE1_sub_a,BASE1_sub_b)
					\tkzDrawSegment(BASE1_sub_a,BASE1_sub_d)
					\path[fill=gray!30] (BASE1_sub_b)--(BASE1_inner_b)--(BASE1_inner_a)--(BASE1_sub_a)--cycle;
					\path[fill=gray!30] (BASE1_sub_d)--(BASE1_inner_d)--(BASE1_inner_a)--(BASE1_sub_a)--cycle;
					\tkzDrawSegment(BASE1_sub_a,BASE1_a)

					%BASE 2
					\coordinate (BASE2_sup_a) at (-1.5,-1.5,1.1);
					\coordinate (BASE2_sup_b) at (1.5,-1.5,1.1);
					\coordinate (BASE2_sup_c) at (1.5,1.5,1.1);
					\coordinate (BASE2_sup_d) at (-1.5,1.5,1.1);
					\coordinate (BASE2_a) at (-1.5,-1.5,1.0);
					\coordinate (BASE2_b) at (1.5,-1.5,1.0);
					\coordinate (BASE2_c) at (1.5,1.5,1.0);
					\coordinate (BASE2_d) at (-1.5,1.5,1.0);
					\coordinate (BASE2_inner_a) at (-1.4,-1.4,1.1);
					\coordinate (BASE2_inner_b) at (1.4,-1.4,1.1);
					\coordinate (BASE2_inner_c) at (1.4,1.4,1.1);
					\coordinate (BASE2_inner_d) at (-1.4,1.4,1.1);
					\tkzDrawSegment(BASE2_a,BASE2_b)
					\tkzDrawSegment(BASE2_a,BASE2_d)
					\tkzDrawSegment(BASE2_sup_a,BASE2_sup_b)
					\tkzDrawSegment(BASE2_sup_a,BASE2_sup_d)
					\tkzDrawSegment(BASE2_sup_a,BASE2_a)

					\tkzDrawSegment(BASE1_a,BASE2_a)

					% DISCO (ENGRENAGEM) NO EIXO VERTICAL
					\addplot3+[domain=pi:2*pi,samples=200,samples y=0,no marks, black, fill=gray!50]
						({\r*cos(deg(x))}, {\r*sin(deg(x))}, {0});

					%LIMITES DO DESENHO 
					\draw[thin, dashed] (0,0,0)--(0,0,5);
					\draw[thin, dashed] (-2,0,0)--(2,0,0);

					%EIXO VERTICAL
					\def\altEixo{2}
					\draw[very thick] (0,0,-0.8)--(0,0,\altEixo);

					% DISCO (ENGRENAGEM) NO EIXO VERTICAL
					\addplot3+[domain=0:pi,samples=200,samples y=0,no marks, black, fill=gray!50]
						({\r*cos(deg(x))}, {\r*sin(deg(x))}, {0});

					%"GARFO"
					\def\altTotal{3.5}
					\draw[very thick] (-1,0,\altEixo)--(1,0,\altEixo) ;
					\draw[very thick] (-1,0,\altEixo)--(-1,0,\altTotal);
					\draw[very thick] (1,0,\altEixo)--(1,0,\altTotal);

					% EIXO VERTICAL SOB O ESPELHO
					\draw[->, >=stealth, dashed, very thin] (0,0,\altEixo)--(0,0,\altEixo + 1.5) ;

					% EIXO DE ROTAÇÃO HORIZONTAL
					\draw[thick] (-1,0,\altTotal)--(1,0,\altTotal) ;

					\def\ri{30}
					\FPeval\angTheta{0+\ri*pi/180}
					\FPeval\senTheta{sin(\angTheta)}
					\FPeval\cosTheta{cos(\angTheta)}

					%					(y   , x  , z    )
					\coordinate (A) at (-0.95 ,1.5*\cosTheta,\altTotal - 1.5*\senTheta);
					\coordinate (B) at (-0.95 ,-1.5*\cosTheta,\altTotal + 1.5*\senTheta);
					\coordinate (C) at (0.95 ,-1.5*\cosTheta,\altTotal + 1.5*\senTheta);
					\coordinate (D) at (0.95 ,1.5*\cosTheta,\altTotal - 1.5*\senTheta);

					\draw[thin, fill=blue!10] (A)--(B)--(C)--(D)--cycle;

					% DISCO (ENGRENAGEM) NO EIXO HORIZONTAL
					\def\raioSup{0.6}
					\addplot3+[domain=0:2*pi,samples=200,samples y=0,no marks, black, fill=gray!50]
						({1.2}, {\raioSup*sin(deg(x))}, {\raioSup*cos(deg(x)) + \altTotal} );

					% EIXO DE ROTAÇÃO HORIZONTAL (PARTE VISÍVEL)
					\draw[thick] (1.2,0,\altTotal)--(1.3,0,\altTotal) ;

					% EIXOS HORIZONTAIS
					\draw[->, >=stealth, dashed, very thin, color=red] (-2,0,0)--(2,0,0) ;
					\draw[->, >=stealth, dashed, very thin, color=blue] (0,-2,0)--(0,2,0) ;

					%EIXO VERTICAL SOBRE O ESPELHO
					\draw[->, >=stealth, dashed, very thin] (0,0,\altEixo+1.5)--(0,0,5) ;

					\coordinate (BASE1_center_ad) at (-1.5,0,-0.6);
					\coordinate (BASE1_center_ab) at (1.5,0,-0.6);
					\coordinate (BASE1_sub_center_ad) at (-1.5,0,-0.7);
					\coordinate (BASE1_sub_center_ab) at (1.5,0,-0.7);
					\coordinate (BASE1_center) at (0,0,-0.6);
					\coordinate (BASE1_sub_center) at (0,0,-0.7);
					\path[draw, fill=white] (BASE1_sub_center)--(BASE1_center)--(BASE1_center_ab)--(BASE1_sub_center_ab)--cycle;
					\path[draw, fill=white] (BASE1_sub_center)--(BASE1_center)--(BASE1_center_ad)--(BASE1_sub_center_ad)--cycle;

					\path[draw, fill=white] (BASE1_c)--(BASE1_sub_c)--(BASE1_sub_d) --(BASE1_d)--cycle;
					\path[draw, fill=white] (BASE1_c)--(BASE1_sub_c)--(BASE1_sub_b) --(BASE1_b)--cycle;

					\coordinate (BASE2_center_ad) at (0,-1.5,1.0);
					\coordinate (BASE2_center_ab) at (0,1.5,1.0);
					\coordinate (BASE2_sub_center_ad) at (0,-1.5,1.1);
					\coordinate (BASE2_sub_center_ab) at (0,1.5,1.1);
					\coordinate (BASE2_center) at (0,0,1.0);
					\coordinate (BASE2_sub_center) at (0,0,1.1);
					\path[draw, fill=white] (BASE2_sub_center)--(BASE2_center)--(BASE2_center_ab)--(BASE2_sub_center_ab)--cycle;
					\path[draw, fill=white] (BASE2_sub_center)--(BASE2_center)--(BASE2_center_ad)--(BASE2_sub_center_ad)--cycle;

					\path[draw, fill=white] (BASE2_c)--(BASE2_sup_c)--(BASE2_sup_d) --(BASE2_d)--cycle;
					\path[draw, fill=white] (BASE2_c)--(BASE2_sup_c)--(BASE2_sup_b) --(BASE2_b)--cycle;

					% \path[draw, fill=gray!20] (BASE2_sup_b)--(BASE2_inner_b)--(BASE2_inner_a)--(BASE2_sup_a)--cycle;
					% \path[draw, fill=gray!20] (BASE2_sup_b)--(BASE2_inner_b)--(BASE2_inner_c)--(BASE2_sup_c)--cycle;
					% \path[draw, fill=gray!20] (BASE2_sup_d)--(BASE2_inner_d)--(BASE2_inner_a)--(BASE2_sup_a)--cycle;
					% \path[draw, fill=gray!20] (BASE2_sup_d)--(BASE2_inner_d)--(BASE2_inner_c)--(BASE2_sup_c)--cycle;

					\tkzDrawSegment[thin](BASE1_b,BASE2_b)
					\tkzDrawSegment[thin](BASE1_c,BASE2_c)
					\tkzDrawSegment[thin](BASE1_d,BASE2_d)


				\end{axis}
			\end{tikzpicture}
			\caption{Heliostato --- detalhes da base}%
		\end{figure}

\end{document}
