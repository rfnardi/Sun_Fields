\documentclass[12pt,a4paper]{article}
\usepackage[a4paper]{geometry}
\geometry{verbose,tmargin=1.5cm,bmargin=2cm,lmargin=2cm,rmargin=2cm}
\usepackage{amsmath,amsfonts,amssymb,amsthm,dsfont,mathtools}
\usepackage{cancel}
\usepackage{setspace}
\usepackage{booktabs}
\usepackage{hyperref}
\usepackage{graphicx}
\usepackage{bookmark}
\usepackage{verbatim}
\onehalfspacing
\usepackage[brazilian]{babel}
\usepackage{tikz, pgfplots}
\usepackage{tikz-3dplot}
\usepackage{tkz-fct, tkz-base, tkz-euclide} %tkz-fct chama gnuplot --> precisa compilar com a opção -shell-escape depois de pdflatex
\usepackage{animate}
\pgfplotsset{width=10cm,compat=1.18}
\usetikzlibrary{calc,angles,quotes,intersections}
\usetikzlibrary{decorations.pathmorphing}
\DeclareMathOperator{\sen}{sen}

\begin{document}
	{\bf Problema:} determinar a posição do sol como função do dia do ano, da hora local e da latitude.

	\begin{figure}[!ht]
		\centering
		\begin{animateinline}[poster=first,controls={play,stop,step}]{8}
		\multiframe{50}{ri=0+1}
		{
			\begin{tikzpicture}[scale=1.3]
				% \useasboundingbox (-1.6, 0) rectangle (8, 8);
				\begin{axis}[view={-20+\ri}{-20}, % {angulo azimutal}{angulo zenital}
					zmax=3,
					xmax=3,
					ymax=3,
					% height=10cm,
					xtick=\empty,
					ytick=\empty,
					ztick=\empty,
					clip=false,
					hide axis
					]
					\def\r{1}

					\addplot3+[domain=pi:2*pi,samples=200,samples y=0,no marks, black, fill=gray!50]
						({\r*cos(deg(x))}, {\r*sin(deg(x))}, {0});

					%EIXO VERTICAL
					\def\altEixo{2}
					\draw[very thick] (0,0,-0.8)--(0,0,\altEixo) ;

					\addplot3+[domain=0:pi,samples=200,samples y=0,no marks, black, fill=gray!50]
						({\r*cos(deg(x))}, {\r*sin(deg(x))}, {0});

					%"GARFO"
					\def\altTotal{3.5}
					\draw[very thick] (-1,0,\altEixo)--(1,0,\altEixo) ;
					\draw[very thick] (-1,0,\altEixo)--(-1,0,\altTotal) ;
					\draw[very thick] (1,0,\altEixo)--(1,0,\altTotal) ;

					\FPeval\angZen{0-10*pi/180}
					\FPeval\senZen{sin(\angZen)}
					\FPeval\cosZen{cos(\angZen)}
					\draw[very thin, fill=blue!10] (-0.95,\cosZen,\altTotal+\senZen)--(-0.95,-\cosZen,\altTotal-\senZen)--(0.95,-\cosZen,\altTotal-\senZen)--(0.95,\cosZen,\altTotal+\senZen)--cycle;

					% EIXO HORIZONTAL
					\draw[dashed, very thin] (-1,0,\altTotal)--(1,0,\altTotal) ;

				\end{axis}
			\end{tikzpicture}
		}
		\end{animateinline}
		\caption{Sistema de coordenadas horizontal}%
	\end{figure}


	\begin{figure}[!ht]
		\centering
		\begin{animateinline}[poster=first,controls={play,stop,step}]{8}
			\multiframe{50}{ri=0+1}
			{
				\begin{tikzpicture}[transform shape, scale=1.3]
				% \useasboundingbox (-8, -8) rectangle (8, 8); %not working
					\begin{axis}[view={-20}{-20}, % {angulo azimutal}{angulo zenital}
						zmax=3,
						xmax=3,
						ymax=3,
						% height=10cm,
						xtick=\empty,
						ytick=\empty,
						ztick=\empty,
						clip=false,
						hide axis
						]
						\def\r{1}

						\addplot3+[domain=pi:2*pi,samples=200,samples y=0,no marks, black, fill=gray!50]
							({\r*cos(deg(x))}, {\r*sin(deg(x))}, {0});

						%LIMITES DO DESENHO 
						\draw[fill=black] (0,0,5) circle[radius=1pt];
						\draw[thin, dashed] (-2,0,0)--(2,0,0);
						\draw[fill=black] (-2,0,0) circle[radius=1pt];
						\draw[fill=black] (2,0,0) circle[radius=1pt];

						%EIXO VERTICAL
						\def\altEixo{2}
						\draw[very thick] (0,0,-0.8)--(0,0,\altEixo);

						\addplot3+[domain=0:pi,samples=200,samples y=0,no marks, black, fill=gray!50]
							({\r*cos(deg(x))}, {\r*sin(deg(x))}, {0});

						%"GARFO"
						\def\altTotal{3.5}
						\draw[very thick] (-1,0,\altEixo)--(1,0,\altEixo) ;
						\draw[very thick] (-1,0,\altEixo)--(-1,0,\altTotal);
						\draw[very thick] (1,0,\altEixo)--(1,0,\altTotal);

						\FPeval\angZen{0-(10+(\ri))*pi/180}
						\FPeval\senZen{sin(\angZen)}
						\FPeval\cosZen{cos(\angZen)}
						\draw[very thin, fill=blue!10] (-0.95,\cosZen,\altTotal+\senZen)--(-0.95,-\cosZen,\altTotal-\senZen)--(0.95,-\cosZen,\altTotal-\senZen)--(0.95,\cosZen,\altTotal+\senZen)--cycle;

						% EIXO HORIZONTAL
						\draw[dashed, very thin] (-1,0,\altTotal)--(1,0,\altTotal) ;
					\end{axis}
				\end{tikzpicture}
			}
		\end{animateinline}
		\caption{Sistema de coordenadas horizontal}%
		\end{figure}
	Queremos descrever o vetor $\hat{s}$ ao longo do ano (NDA), ao longo da
	hora local e também sua dependência com a latitude ($\phi$). Para isso
	precisamos expressar como os ângulos $\color{red}{z}$ ({\bf distância zenital}) e
	$\color{blue}{A}$ ({\bf azimute}) dependem dessas variáveis.
	
\end{document}
