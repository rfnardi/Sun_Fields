\documentclass[12pt,a4paper]{article}
\usepackage[a4paper]{geometry}
\geometry{verbose,tmargin=1.5cm,bmargin=2cm,lmargin=2cm,rmargin=2cm}
\usepackage{amsmath,amsfonts,amssymb,amsthm,dsfont,mathtools}
\usepackage{cancel}
\usepackage{setspace}
\usepackage{booktabs}
\usepackage{hyperref}
\usepackage{graphicx}
\usepackage{bookmark}
\usepackage{verbatim}
\onehalfspacing
\usepackage[brazilian]{babel}
\usepackage{tikz, pgfplots}
\usepackage{tikz-3dplot}
\usepackage{tkz-fct, tkz-base, tkz-euclide} %tkz-fct chama gnuplot --> precisa compilar com a opção -shell-escape depois de pdflatex
\usepackage{animate}
\pgfplotsset{width=10cm,compat=1.18}
\usetikzlibrary{calc,angles,quotes,intersections}
\usetikzlibrary{decorations.pathmorphing}
\DeclareMathOperator{\sen}{sen}

\begin{document}

\begin{center} {\bf \Large Movimentos do heliostato} \end{center}

	\begin{figure}[!ht]
		\centering
		\begin{animateinline}[poster=first,controls={play,stop,step}]{8}
			\multiframe{50}{ri=0+2} {
			\begin{tikzpicture}[scale=1.3]
				% \useasboundingbox (-1.6, 0) rectangle (8, 8);
				\begin{axis}[view={-20}{-20}, % {angulo azimutal}{angulo zenital}
					zmax=3, xmax=3, ymax=3,
					% height=10cm,
					xtick=\empty, ytick=\empty, ztick=\empty,
					clip=false, hide axis ]
					\def\r{1}

					\addplot3+[domain=pi:2*pi,samples=200,samples y=0,no marks, black, fill=gray!50]
						({\r*cos(deg(x))}, {\r*sin(deg(x))}, {0});

					%EIXO VERTICAL
					\def\altEixo{2}
					\draw[very thick] (0,0,-0.8)--(0,0,\altEixo) ;

					\addplot3+[domain=0:pi,samples=200,samples y=0,no marks, black, fill=gray!50]
						({\r*cos(deg(x))}, {\r*sin(deg(x))}, {0});

					\draw[->, >=stealth, dashed, very thin] (0,0,\altEixo)--(0,0,\altEixo + 1.5) ;

					%"GARFO"
					\def\altTotal{3.5}
					\FPeval\angPhi{(\ri)*pi/180}
					\FPeval\senPhi{sin(\angPhi)}
					\FPeval\cosPhi{cos(\angPhi)}
					\draw[very thick] (-\cosPhi,\senPhi,\altEixo)--(\cosPhi,-\senPhi,\altEixo) ;
					\draw[very thick] (-\cosPhi,\senPhi,\altEixo)--(-\cosPhi,\senPhi,\altTotal) ;
					\draw[very thick] (\cosPhi,-\senPhi,\altEixo)--(\cosPhi,-\senPhi,\altTotal) ;

					\FPeval\angTheta{0+20*pi/180}
					\FPeval\senTheta{sin(\angTheta)}
					\FPeval\cosTheta{cos(\angTheta)}


					%					(y   , x  , z    )
					\coordinate (A) at (-0.95*\cosPhi +\senPhi*1.5*\cosTheta,1.5*\cosTheta*\cosPhi +0.95*\senPhi,\altTotal - 1.5*\senTheta);
					\coordinate (B) at (-0.95*\cosPhi -\senPhi*1.5*\cosTheta,-1.5*\cosTheta*\cosPhi +0.95*\senPhi,\altTotal + 1.5*\senTheta);
					\coordinate (C) at (0.95*\cosPhi -\senPhi*1.5*\cosTheta,-1.5*\cosTheta*\cosPhi - 0.95*\senPhi,\altTotal + 1.5*\senTheta);
					\coordinate (D) at (0.95*\cosPhi +\senPhi*1.5*\cosTheta,1.5*\cosTheta*\cosPhi - 0.95*\senPhi,\altTotal - 1.5*\senTheta);

					\draw[thin, fill=blue!10] (A)--(B)--(C)--(D)--cycle;

					% EIXOS HORIZONTAIS
					\draw[->, >=stealth, dashed, very thin, color=red] (-2,0,0)--(2,0,0) ;
					\draw[->, >=stealth, dashed, very thin, color=blue] (0,-2,0)--(0,2,0) ;

					\draw[->, >=stealth, dashed, very thin] (0,0,\altEixo+1.5)--(0,0,5) ;

				\end{axis}
			\end{tikzpicture} }
		\end{animateinline}
		\caption{Heliostato --- Movimento no eixo vertical}%
	\end{figure}


	\begin{figure}[!ht]
		\centering
		\begin{animateinline}[poster=first,controls={play,stop,step}]{8}
			\multiframe{20}{ri=0+2}
			{
				\begin{tikzpicture}[transform shape, scale=1.3]
				% \useasboundingbox (-8, -8) rectangle (8, 8); %not working
					\begin{axis}[view={-20}{-20}, % {angulo azimutal}{angulo zenital}
						zmax=3,
						xmax=3,
						ymax=3,
						% height=10cm,
						xtick=\empty,
						ytick=\empty,
						ztick=\empty,
						clip=false,
						hide axis
						]
						\def\r{1}

						\addplot3+[domain=pi:2*pi,samples=200,samples y=0,no marks, black, fill=gray!50]
							({\r*cos(deg(x))}, {\r*sin(deg(x))}, {0});

						%LIMITES DO DESENHO 
						\draw[thin, dashed] (0,0,0)--(0,0,5);
						\draw[thin, dashed] (-2,0,0)--(2,0,0);

						%EIXO VERTICAL
						\def\altEixo{2}
						\draw[very thick] (0,0,-0.8)--(0,0,\altEixo);

						\addplot3+[domain=0:pi,samples=200,samples y=0,no marks, black, fill=gray!50]
							({\r*cos(deg(x))}, {\r*sin(deg(x))}, {0});

						%"GARFO"
						\def\altTotal{3.5}
						\draw[very thick] (-1,0,\altEixo)--(1,0,\altEixo) ;
						\draw[very thick] (-1,0,\altEixo)--(-1,0,\altTotal);
						\draw[very thick] (1,0,\altEixo)--(1,0,\altTotal);

						% EIXO VERTICAL SOB O ESPELHO
						\draw[->, >=stealth, dashed, very thin] (0,0,\altEixo)--(0,0,\altEixo + 1.5) ;

						\FPeval\angTheta{0+\ri*pi/180}
						\FPeval\senTheta{sin(\angTheta)}
						\FPeval\cosTheta{cos(\angTheta)}

						%					(y   , x  , z    )
						\coordinate (A) at (-0.95 ,1.5*\cosTheta,\altTotal - 1.5*\senTheta);
						\coordinate (B) at (-0.95 ,-1.5*\cosTheta,\altTotal + 1.5*\senTheta);
						\coordinate (C) at (0.95 ,-1.5*\cosTheta,\altTotal + 1.5*\senTheta);
						\coordinate (D) at (0.95 ,1.5*\cosTheta,\altTotal - 1.5*\senTheta);

						\draw[thin, fill=blue!10] (A)--(B)--(C)--(D)--cycle;

						% EIXOS HORIZONTAIS
						\draw[->, >=stealth, dashed, very thin, color=red] (-2,0,0)--(2,0,0) ;
						\draw[->, >=stealth, dashed, very thin, color=blue] (0,-2,0)--(0,2,0) ;

						%EIXO VERTICAL SOBRE O ESPELHO
						\draw[->, >=stealth, dashed, very thin] (0,0,\altEixo+1.5)--(0,0,5) ;

					\end{axis}
				\end{tikzpicture}
			}
		\end{animateinline}
		\caption{Heliostato --- movimento do eixo horizontal}%
		\end{figure}

\end{document}
