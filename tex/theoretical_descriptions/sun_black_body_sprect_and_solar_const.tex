\documentclass[12pt,a4paper]{article}
\usepackage[a4paper]{geometry}
\geometry{verbose,tmargin=1.5cm,bmargin=2cm,lmargin=2cm,rmargin=2cm}
\usepackage{amsmath,amsfonts,amssymb,amsthm,dsfont,mathtools}
\usepackage{cancel}
\usepackage{setspace}
\usepackage{booktabs}
\usepackage{hyperref}
\usepackage{graphicx}
\usepackage{bookmark}
\usepackage{verbatim}
\onehalfspacing
\usepackage[brazilian]{babel}
\usepackage{tikz, pgfplots}
\usepackage{tikz-3dplot}
\usepackage{tkz-fct, tkz-base, tkz-euclide} %tkz-fct chama gnuplot --> precisa compilar com a opção -shell-escape depois de pdflatex
\usepackage{animate}
\pgfplotsset{width=10cm,compat=1.17}
\usetikzlibrary{calc,angles,quotes,intersections}
\usetikzlibrary{decorations.pathmorphing}
\DeclareMathOperator{\sen}{sen}

\begin{document}

	\begin{center} {\bf \large Relação entre espectro de corpo negro e constante solar.} \end{center}

	Vamos agora organizar as idéias para entender qual é a função $P(\lambda)$
	que integrada em $\lambda$ produz a constante solar.

	A densidade de energia entre $\lambda$ e $\lambda + \d \lambda$
	de um corpo negro é dada pela lei de Planck:
	\begin{equation}
		I(\lambda) = \frac{}{}
	\end{equation}


\end{document}
