\documentclass[12pt,a4paper]{article}

\usepackage[a4paper]{geometry}
\geometry{verbose,tmargin=1.5cm,bmargin=2cm,lmargin=2cm,rmargin=2cm}
\usepackage{amsmath,amsfonts,amssymb,amsthm,dsfont,mathtools}
\usepackage{cancel}
\usepackage{setspace}
\usepackage{booktabs}
\usepackage{hyperref}
\usepackage{graphicx}
\usepackage{bookmark}
\usepackage{verbatim}
\usepackage{xcolor}
\onehalfspacing
\usepackage[brazilian]{babel}

\usepackage{tikz, pgfplots}
\usepackage{tkz-fct, tkz-base, tkz-euclide} %tkz-fct chama gnuplot --> precisa compilar com a opção -shell-escape depois de pdflatex
\pgfplotsset{width=10cm,compat=1.17}
\usetikzlibrary{calc,angles,quotes,intersections}
\usetikzlibrary{decorations.pathmorphing}

\DeclareMathOperator{\sen}{sen}

\begin{document}
	
	{\large \bf Problema:} 

	\section{Determinação da distância percorrida ${\color{red} d}$ por um raio de luz
	do sol dentro da atmosfera terrestre como função do ângulo zenital $\theta_z$ --- sem correção por refração.}

	\bigskip

	\begin{figure}[htpb]
		\centering
		\begin{tikzpicture}[scale=1, transform shape]
			\def\squareSize{4}
			\tkzInit[xmin=-\squareSize,xmax=+\squareSize,ymin=-\squareSize,ymax=\squareSize]
			\tkzDefPoint(0,0){O}
			\tkzDefPoint(0,2.5){A}
			\tkzDefPoint(0,4){B}
			\tkzDefPoint(55:4){C} \tkzDefPoint(55:0.7){c}
			\tkzDefPoint(55:2.5){E}
			\tkzDrawCircle(O,A)
			\tkzDrawCircle(O,B)
			\tkzDrawSegment[dashed](O,B)
			\tkzDrawSegment[dashed](O,C)
			\tkzDrawSegment[red](A,C)
			\tkzDefPointBy[projection=onto O--B](C) \tkzGetPoint{D}
			\tkzDrawSegment[dashed](D,C)

			\tkzDefMidPoint(O,E) \tkzGetPoint{Rt} \tkzLabelPoint[below right](Rt){$R_T$}
			\tkzDefMidPoint(E,C) \tkzGetPoint{H} \tkzLabelPoint[below right](H){$H$}
			\tkzDefMidPoint(A,D) \tkzGetPoint{h} \tkzLabelPoint[left](h){$h$}
			\tkzDefMidPoint(C,D) \tkzGetPoint{l} \tkzLabelPoint[above](l){$l$}
			\tkzDefMidPoint(A,C) \tkzGetPoint{d} \tkzLabelPoint[below, red](d){$d$}

			\tkzDrawArc[black](O,c)(A) \tkzLabelPoint[above]({70.5:0.7}){$\alpha$}

			\tkzDefPointWith[linear, K=0.15](A,C) \tkzGetPoint{zen_angle}
			\tkzDrawArc[black](A,zen_angle)(B) 

			\tkzDefPointBy[rotation=center A angle 20](zen_angle) 
			\tkzLabelPoint[above](tkzPointResult){$\theta_z$}
	\end{tikzpicture}
		\caption{{\color{red} Caminho} do raio de luz na atmosfera.}%
		\label{fig:caminho_luz_na_atmosfera}
	\end{figure}

	\bigskip

	Da figura (\ref{fig:caminho_luz_na_atmosfera}) extraímos as seguintes expressões trigonométricas:

	\begin{equation}
		d^2 = h^2 + l^2 \ ;\ \text{(Pitágoras)} \label{eq:pit_1},
	\end{equation}

	\begin{equation}
		{(H + R_T)}^2 = {(h + R_T)}^2 + l^2 \ ;\ \text{(Pitágoras)}  \label{eq:pit_2},
	\end{equation}
	e
	\begin{equation}
		\cos(\theta_z) = \frac{h}{d}   \label{eq:cos_zen}.
	\end{equation}

	Tomamos (\ref{eq:pit_2})  $-$ (\ref{eq:pit_1}):
	\begin{eqnarray}
		{(H + R_T)}^2 - d^2 &=& {(h + R_T)}^2 - h^2 \\
		H^2 + 2 H R_T + \bcancel{{R_T}^2} - d^2 &=& \cancel{h^2}  + 2 h R_T + \bcancel{{R_T}^2} - \cancel{h^2} 
	\end{eqnarray}

	\begin{equation}
		 d^2 + 2 h R_T - (H^2 + 2 H R_T) = 0
	\end{equation}
	e agora usamos (\ref{eq:cos_zen}):
	\begin{equation}
		 d^2 + 2 (d\cos(\theta_z)) R_T - (H^2 + 2 H R_T) = 0. \label{eq:2nd_order_equation_for_d}
	\end{equation}

	Vemos que (\ref{eq:2nd_order_equation_for_d}) é uma equação de segunda ordem para $d$. Logo, 
	\begin{equation}
		d = \frac{-2R_T\cos(\theta_z) \pm \sqrt{ {(2R_T\cos(\theta_z))}^2-4\cdot 1 \cdot (-(H^2 + 2H R_T)) }}{2}	
	\end{equation}
	donde
	\begin{equation}
		d = \frac{-\cancel{2}R_T\cos(\theta_z) \pm \cancel{2}R_T\cos(\theta_z)\sqrt{ 1 +  \frac{H(H + 2R_T)}{{R_T}^2\cos^2(\theta_z)} }}{\cancel{2}}	.
		\label{eq:almost_there}
	\end{equation}

	Usamos agora o fato de que dado que o raio da Terra ($\sim 6500 km$) é
	muito maior do que a espessura da atmosfera terrestre ($\sim 12 km$), o que
	faz com que a fração que aparece dentro da raiz quadrada em
	(\ref{eq:almost_there}) tenha valor inferior a $1$ o que por sua vez nos
	permite lançar mão da aproximação binomial em primeira ordem:
	\begin{eqnarray}
		d &\approx& R_T\cos(\theta_z)(-1 \pm { 1 + \frac{H(H + 2R_T)}{2{R_T}^2\cos^2(\theta_z)}}) \label{eq:almost_there1}\\
		  &\approx& R_T\cos(\theta_z)(\cancel{-1} + \cancel{1} + \frac{H(H + 2R_T)}{2{R_T}^2\cos^2(\theta_z)}) \label{eq:almost_there2}\\
		  &\approx& \cancel{R_T\cos(\theta_z)}(\frac{H(H + 2R_T)}{2{R_T}^{\cancel{2}}\cos^{\cancel{2}}(\theta_z)})
	\end{eqnarray}
	Onde de (\ref{eq:almost_there1}) para (\ref{eq:almost_there2}) restringimos
	apenas ao sinal positivo já que, sendo $d$ uma distância não pode ser
	negativa.

	Finalmente temos a distância percorrida pelo raio de luz do sol
	no interior da atmosfera até atingir a superfície do planeta à um ângulo
	zenital $\theta_z$:
	\begin{equation}
		 \boxed{d \approx \frac{H(H + 2R_T)}{2{R_T}\cos(\theta_z)}.} \label{eq:straight_optical_path}
	\end{equation}

	\section{Cálculo da massa óptica relativa}

	A equação (\ref{eq:straight_optical_path}) dá uma
	aproximação~\cite{muhamad_iqbal} com erro inferior a $0,25\%$ para
	$\theta_z=60^{\circ}$ que aumenta para $10\%$ para ângulos superiores a
	$85^{\circ}$. 

	Faremos aqui uma primeira abordagem com a correção na intensidade da
	radiação dada pela equação (\ref{eq:straight_optical_path}) e em seguida
	consideraremos refinamentos deste resultado. Na prática esta aproximação
	assume que a densidade da atmosfera é homogênea. Sob esta hipótese ocorre
	que a intensidade da radiação que chega ao solo sofre uma correção
	proporcional a este caminho posto que a quantidade de energia liberada por
	um feixe de radiação num meio é proporcional à quantidade de massa presente
	no caminho do feixe de radiação.

	Usa-se na literatura o termo \emph{massa óptica relativa} $m_r(\theta_z)$ para
	designar a razão entre a quantidade de massa atmosférica $m_0$ atravessada
	pelo raio luminoso no zênite e a quantidade de massa de atmosférica $m(\theta_z)$
	atravessada à um ângulo zenital $\color{red}{\theta_z}$:
	\begin{equation}
		m_r (\theta_z) = \frac{m_0}{m(\theta_z)}.
	\end{equation}

	Como estamos trabalhando sob a hipótese de que a massa é proporcional ao
	comprimento do caminho óptico temos que (\ref{eq:straight_optical_path})
	incorre em 
	\begin{equation}
		m_r (\theta_z) = \frac{1}{\cos(\theta_z)}. \label{eq:0-th_order_optical_mass_correction}
	\end{equation}
	que será tomada como nosso termo de ordem zero na correção do fator de
	absorção sofrido pelo feixe de radiação.

	\subsection{Correção para a luz visível pela massa de ar seco }

	Para comprimentos de onda na faixa visível ($\sim 0.7\mu m$) Kasten 
	apresenta a seguinte correção para $m_r (\theta_z)$:
	\begin{equation}
		m_r (\theta_z) = {\left(\cos(\theta_z) + 0.15{(93,885-\theta_z)}^{-1,253}\right)}^{-1}. \label{eq:dry_air_corr_to_opt_mass}
	\end{equation}

	Confrontada com valores medidos experimentalmente, a expressão
	(\ref{eq:dry_air_corr_to_opt_mass}) fornece dados com erro na
	inferior a $0,1\%$ para medidas do ângulo zênital $\theta_z$ variando desde $0$ a
	$86^{\circ}$. O pior valor ocorre em $\theta_z = 89,5^{\circ}$ quando o
	erro atinge $1,25\%$.

	\subsection{Correção para a luz visível pela massa de vapor de água }

	A correção dada pela interação da radiação com vapor de água
	ocorre predominantemente nas camadas mais baixas da atmosfera.
	Assim, também Kasten propõe a seguinte fórmula para a respectiva
	massa optica relativa:
	\begin{equation}
		m_r (\theta_z) = {\left(\cos(\theta_z) + 0.00548{(92,650-\theta_z)}^{-1,452}\right)}^{-1}. \label{eq:w_vapor_corr_to_opt_mass}
	\end{equation}

	\section{Correção da intensidade da radiação devido à interação com os componentes atmosféricos}

	De acordo com a lei de Bourguer~\cite{muhamad_iqbal}, a intensidade do 
	feixe de radiação monocromática (medida perpendicularmente à direção de
	propagação do feixe) sofre uma correção de acordo com a seguinte formula:
	\begin{equation}
		I_{\lambda} = I_{\lambda 0}e^{-k_{\lambda}m_{\lambda}} \label{eq:bourguers_law}
	\end{equation}
	onde $k_\lambda$ é chamado de \emph{fator de extinção} e $m_\lambda$ é a
	massa óptica com a qual o feixe interage. Este fator exponencial 
	\begin{equation}
		\tau = e^{-k_{\lambda}m_{\lambda}}
	\end{equation}é chamado de \emph{transmitância} $\tau$ da radiação.

	À medida que a radiação penetra na atmosfera, sua intensidade sofre
	correção devido a vários processos de absorção e espalhamento. Todos eles
	se combinam aditivamente:
	\begin{equation}
		k_{\lambda}m_{\lambda} = \sum_{i=1}^{j} k_{\lambda i}m_{\lambda i}
	\end{equation}
	onde $i$ indica um dos $j$ processos pelos quais ocorre absorção.
	O que mostra que $\tau$ pode ser escrito como um produtório onde cada fator 
	dá conta de um efeito responsável por atenuar a intensidade do feixe.
	Vamos agora listar os principais fatores de extinção e suas respectivas 
	transmitâncias.

	O fator de extinção devido ao espalhamento da luz de comprimento de onda $\lambda$ 
	por partículas de diâmetro $d$ tal que $\lambda > 10 d$ é dado pela teoria de Rayleigh.
	Para ar seco pode ser aproximado por 
	\begin{equation}
		k_{\lambda,Ray} \approx 0,0087352\lambda^{-4,08}. \label{eq:dry_air_extinc_fac_scat_rayl}
	\end{equation}

	Temos assim uma transmitância devida ao espelhamento de luz por ar seco dada por
	\begin{equation}
		\tau_{\lambda,Ray} = e^{0,0087352\lambda^{-4,08}m_a} \label{eq:dry_air_rayl_scat_transm}
	\end{equation}
	onde $m_a$ é a massa óptica à pressão atmosférica local ($m_a=1$ para a pressão padrão de $1atm$).

	Para partículas com diâmetro maior do que um décimo do comprimento de onda
	o fator de espalhamento é calculado pela teoria de Mie para o espalhamento
	de luz. Neste intervalo de valores se enquadra o espalhamento de luz por
	partículas de poeira e por gotículas de água.

	O espalhamento por gotículas de água é dado por
	\begin{eqnarray}
		k_{\lambda,wd} \approx 0,008635\lambda^{-2},\\
		\nonumber \\
		\tau_{\lambda,wd} = e^{0,008635\lambda^{-2}w m_r}  \label{eq:wat_droplets_scat_transm}
	\end{eqnarray}
	onde $w$ é o tamanho da gotícula de água e $m_r$ é a massa óptica
	dada pela equação (\ref{eq:dry_air_corr_to_opt_mass}).

	Já o espalhamento por partículas de poeira é dado por
	\begin{eqnarray}
		k_{\lambda,dust} \approx 0,08128\lambda^{-0,75}.\\
		\nonumber \\
		\tau_{\lambda,dust} = e^{0,08128\lambda^{-0,75}\frac{d}{800}m_a}   \label{eq:dust_scat_transm}
	\end{eqnarray}
	onde $d$ é a quantidade de partículas de poeira por $cm^3$ e $m_a$ é a 
	massa óptica à pressão medida.

	Com essas 3 contribuições podemos expressar a transmitância global da
	radiação direta para luz monocromática devido ao espalhamento $\tau_{\lambda,S}$:
	\begin{equation}%EQUAÇÃO 6.7.2 DO MUHAMAD IQBAL
		\tau_{\lambda,S} = e^{-(k_{\lambda,Ray} + k_{\lambda,wd}w + k_{\lambda,dust}\frac{d}{800} + )m_{\lambda}}
		\label{eq:global_scatt_transm}
	\end{equation}
	onde usamos $m_\lambda$ como símbolo genérico para a massa óptica.

	Quanto à atenuação devido à absorção, a transmitância decorre de 3
	principais contribuições: mistura de gases ($\tau_{\lambda,g}$),
	vapor de água ($\tau_{\lambda,wv}$) e absorção pela camada de ozônio ($\tau_{\lambda,O}$). 

	As expressões são as seguintes:
	\begin{eqnarray}%EQ 6.13.1 IQBAL
		\tau_{\lambda,g} = e^{-1,41 k_{\lambda,g}m_a{(1+118,93 k_{\lambda,g}m_a)}^{-0,45}}, 
		\label{eq:gas_mix_absp_transm}
	\end{eqnarray}

	\begin{eqnarray}%EQ 6.13.2 IQBAL
		\tau_{\lambda,wv} = e^{-0,2385 k_{\lambda,wv}w m_r{(1+20,07 k_{\lambda,g}w m_r)}^{-0,45}}, 
		\label{eq:wat_v_absp_transm}
	\end{eqnarray}
	e
	\begin{eqnarray}
		\tau_{\lambda,O} = e^{-k_{\lambda,O} l m_r}
	\end{eqnarray}
	onde $k_{\lambda,O}$ é o fator de atenuação para o ozônio, $m_r$ sua
	massa óptica e $l$ é a espessura da camada de ozônio.

	No final do dia, as contribuições de atenuação devido a fenômenos de absorção constroem 
	a transmitância devido a absorção $\tau_{\lambda,A}$:
	\begin{equation}
		\tau_{\lambda,A} = \tau_{\lambda,g}\tau_{\lambda,wv}\tau_{\lambda,O} \label{eq:global_abs_transm}
	\end{equation}

	Com todas essas contribuições estabelecidas podemos escrever a transmitância global 
	da radiação direta de comprimento de onda $\lambda$ como um produto entre a transmitância devido 
	ao espalhamento (\ref{eq:global_scatt_transm}) pela transmitância devido à absorção (\ref{eq:global_abs_transm}):
	\begin{equation}
		\tau_{\lambda} = \tau_{\lambda,S}\tau_{\lambda,A}
	\end{equation}






	\bibliographystyle{ieeetr}
	\bibliography{./refs.bib}

\end{document}
