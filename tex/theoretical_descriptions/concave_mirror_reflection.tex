\documentclass[12pt,a4paper]{article}
\usepackage[a4paper]{geometry}
\geometry{verbose,tmargin=1.5cm,bmargin=2cm,lmargin=2cm,rmargin=2cm}
\usepackage{amsmath,amsfonts,amssymb,amsthm,dsfont,mathtools}
\usepackage{cancel}
\usepackage{setspace}
\usepackage{booktabs}
\usepackage{hyperref}
\usepackage{graphicx}
\usepackage{bookmark}
\usepackage{verbatim}
\onehalfspacing
\usepackage[brazilian]{babel}
\usepackage{tikz, pgfplots}
\usepackage{tikz-3dplot}
\usepackage{tkz-fct, tkz-base, tkz-euclide} %tkz-fct chama gnuplot --> precisa compilar com a opção -shell-escape depois de pdflatex
\usepackage{animate}
\pgfplotsset{width=10cm,compat=1.17}
\usetikzlibrary{calc,angles,quotes,intersections}
\usetikzlibrary{decorations.pathmorphing}
\DeclareMathOperator{\sen}{sen}

\begin{document}
	{\bf Problema:} determinar o ponto focal de um espelho parabólico cuja parábola é dada por $y(x)=ax^2$.

	\begin{figure}[htpb]
		\centering
	\begin{tikzpicture}[scale=1, transform shape]
		\tkzInit[xmin=-5,xmax=5,ymax=6,ymin=-0.5]
		\tkzDrawY[noticks]
		\FPset\a{0.05}
		\tkzFct[color=blue, domain=-5:5, samples=200]{\a*x*x}
		\tkzDefPoint(0,0){O}
		\FPset\sx{0} %THIS GUY --- direção dos raios incidentes
		\FPset\sy{1} %THIS GUY --- direção dos raios incidentes

		\FPeval\normaS{root(2,(((\sx)*(\sx))+((\sy)*(\sy))))}
		\FPeval\SX{(\sx)/\normaS}
		\FPeval\SY{(\sy)/\normaS}
		\tkzDefPoint(\SX,\SY){S}
		\tkzDefPoint(0,5){B}

		\foreach \x in {1,2,...,4}{
			\FPeval\fx{\a*\x*\x}
			\tkzDefPoint(\x,\fx){A\x}
			\FPeval\deriv{2*\a*\x}
			\FPeval\norm{root(2,1+pow(2,\deriv))}
			\tkzDefPoint(1/\norm,\deriv/\norm){v\x}
			\FPeval\nx{(-\deriv)/\norm}
			\FPeval\ny{(1/\norm)}
			\tkzDefPoint(\nx,\ny){n\x}
			\tkzDefPointWith[colinear=at A\x](O,v\x) \tkzGetPoint{V\x}
			\tkzDefPointWith[colinear=at A\x](O,n\x) \tkzGetPoint{N\x}
			\tkzDefPointWith[colinear=at A\x](O,S) \tkzGetPoint{SunAtPoint\x}
			\tkzDrawLine[thin, dashed, add = 0 and 4, color=orange](A\x,SunAtPoint\x)
			\tkzDrawSegment[->,> = stealth](A\x,V\x)
			\tkzDrawSegment[->,> = stealth](A\x,N\x)
			\FPeval\sprod{(((\nx)*(\SX)) + ((\ny)*(\SY)))}
			\FPeval\angulo{2*(arccos(\sprod))}
			\tkzDefPointBy[rotation in rad=center O angle \angulo](S) \tkzGetPoint{S\x}
			\tkzDefPointWith[colinear=at A\x](O,S\x) \tkzGetPoint{R\x}
			\tkzInterLL(A\x,R\x)(O,B) \tkzGetPoint{C\x}
			\tkzDrawSegment[thin, dashed, color=orange](A\x,C\x)
			\tkzDrawSegment[color=red,->,> = stealth](A\x,R\x)
		}
		
	\end{tikzpicture}
	\caption{Foco de um espelho parabólico}%
	\label{fig:espelho_parab}
	\end{figure}
	
	É importante notar que se variarmos a direção dos raios incidentes
	(controlado nas variáveis $sx$ e $sy$ no script da figura acima) os raios
	refletidos deixam de se encontrar no mesmo ponto. Isto implica que não é
	possível usar arranjos parabólicos de espelhos planos num mesmo heliostato
	para focalizar mais mais luz com menor número de heliostatos (e assim
	reduzir a quantidade de partes móveis contribuindo consequentemente para
	redução do custo).

	Uma alternativa para contornar esta dificuldade poderia ser a de usar
	lentes divergentes no ponto focal de cada espelho parabólico que
	paralelizariam os raios num feixe mais concentrado que por sua vez pode ser
	refletido por um segundo espelho plano móvel. Esta solução viabiliza a
	concentração de mais luz (maior potência) com menor quantidade de
	heliostatos mas por outro lado introduz custos adicionais: lente,
	segundo espelho e mais partes móveis. Muito provavelmente o ganho de
	potência não compensa os custos adicionais.

\end{document}
