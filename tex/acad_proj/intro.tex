Inicie dizendo qual é o seu objeto de estudo, o seu tema (definido anteriormente).

O tema já deve trazer, em sua descrição, o problema.

Então, apresente genericamente a gênese do problema, o contexto do problema, sob o ponto de vista sócio-cultural, da história, do Direito, ou de outro aspecto que permita situar o problema que pretende investigar em sua inter-relação com a sociedade.

Nesta etapa não se deve tomar posições sobre o tema, apenas reproduzir a realidade.
