Depois, é preciso se debruçar sobre a justificativa.

Quando for fazer seu projeto de pesquisa, pense nesse aspecto como a parte em que você explica para um investidor porque é um bom negócio investir na sua ideia.

Podem estar envolvidos na Justificativa as possibilidades que o projeto tem para ser desenvolvido levando-se em consideração a sua própria carga de experiências e níveis formativos, que auxiliem demonstrar que você é o pesquisador ideal para desenvolvê-la.

Como a Justificativa nada mais é que “convencer o outro”, é importante o pesquisador colocar-se na posição de alguém alheio à pesquisa para analisar os motivos pelos quais seria levado a ler tal estudo.

Assim, é importante realizar também conexões do seu tema a outras pesquisas, bibliografias, descobertas recentes, em função de que a importância do tema a ser trabalho, cresce á medida que consigamos ligá-lo ao mundo externo.

Seja sucinto: fale apenas o que realmente vai chamar a atenção para seu projeto.

Para complementar a importância, destaque a pergunta a ser respondida pela pesquisa.
