% Inicie dizendo qual é o seu objeto de estudo, o seu tema (definido anteriormente).

% O tema já deve trazer, em sua descrição, o problema.

% Então, apresente genericamente a gênese do problema, o contexto do problema, sob o ponto de vista sócio-cultural, da história, do Direito, ou de outro aspecto que permita situar o problema que pretende investigar em sua inter-relação com a sociedade.

% Nesta etapa não se deve tomar posições sobre o tema, apenas reproduzir a realidade.

Propomos o estudo da captação de energia termossolar no estado da Bahia.
Especificamente, como a potência captada depende de vários parâmetros numa
instalação do tipo torre central [$ref$].

O problema se contextualiza dentro da pauta global de incremento da produção de
energia por fontes alternativas aos combustíveis fósseis, como forma de
contenção do processo de aquecimento globa em que o uso da tecnologia
termossolar figura como uma proposta ainda pouco explorada no cenário
brasileiro.

O problema da produção de energia por fontes renováveis vem sendo objeto de
diversos estudos nas últimas décadas. Chamamos atenção para o estudo
[$ref-atlas_solar$] em que um panorama geral climatológico do estado da bahia
foi conduzido com foco na avaliação de viabilidade econômica levando em conta
custos vigentes no mercado internacional naquele momento concernentes à tecnologia 
de produção de energia elétrica por placas fotovoltaicas.

Consideramos importante mencionar que o referido estudo apenas menciona a
tecnologia termossolar de torre central de forma tangencial, focando na
avaliação da alternativa fotovoltaica --- a mais vigente no mercado e que vem
recebendo substanciais investimentos já há mais de 50 anos [$ref$].

A tecnologia de captação de energia termossolar no modelo de torre central é
apontada no referido estudo como dependente criticamente da precisão dos
aparelhos que realizam a concentração dos raios solares no ponto focal.

Neste sentido, o estudo que vimos propor se insere no contexto de cobrir esta
lacuna apontando ou não a viabilidade desta tecnologia que em última instância
pode vir a subsidiar investimentos estratégicos seja do setor público, seja da
iniciativa privada no território do estado da Bahia.
