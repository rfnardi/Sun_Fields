Consiste em apresentar um resumo do que já foi escrito sobre o tema no referencial teórico.

Pode-se dizer que a fundamentação teórica é uma pesquisa prévia sobre o que já foi escrito sobre o tema sobre o qual pretende estudar.

Mesmo que seja uma pesquisa inédita como em uma tese, por exemplo, a procura destas fontes, documentais ou bibliográficas, é fundamental para que você não proponha uma pesquisa que já está feita.

A citação das principais conclusões a que outros autores chegaram permite destacar a contribuição da pesquisa realizada, além das contradições ou reafirmar comportamentos e atitudes.
