% TEXTO GERADO PELA CHATGPT:

A necessidade de fontes de energia limpa e renovável tem crescido cada vez mais, e a energia solar é uma das principais alternativas para atender essa demanda. Além disso, a produção de tijolos de barro cozido é um processo que consome muita energia, e a utilização de fontes limpas é uma forma de reduzir a pegada de carbono e contribuir para a mitigação do efeito estufa. A utilização de heliostatos com 2 graus de liberdade movimentados por motores DC e controlados por potenciômetros de 10 voltas garantirá a precisão e eficiência do sistema. A instalação de uma torre central de energia solar heliotérmica no sertão da Bahia pode trazer benefícios econômicos e ambientais, como a geração de empregos e a redução da dependência de combustíveis fósseis.

-----------------------------------------


Depois, é preciso se debruçar sobre a justificativa.

Quando for fazer seu projeto de pesquisa, pense nesse aspecto como a parte em que você explica para um investidor porque é um bom negócio investir na sua ideia.

Podem estar envolvidos na Justificativa as possibilidades que o projeto tem para ser desenvolvido levando-se em consideração a sua própria carga de experiências e níveis formativos, que auxiliem demonstrar que você é o pesquisador ideal para desenvolvê-la.

Como a Justificativa nada mais é que “convencer o outro”, é importante o pesquisador colocar-se na posição de alguém alheio à pesquisa para analisar os motivos pelos quais seria levado a ler tal estudo.

Assim, é importante realizar também conexões do seu tema a outras pesquisas, bibliografias, descobertas recentes, em função de que a importância do tema a ser trabalho, cresce á medida que consigamos ligá-lo ao mundo externo.

Seja sucinto: fale apenas o que realmente vai chamar a atenção para seu projeto.

Para complementar a importância, destaque a pergunta a ser respondida pela pesquisa.
