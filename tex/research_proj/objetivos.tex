% TEXTO GERADO PELA CHATGPT:
O objetivo principal deste projeto é desenvolver e testar um protótipo de heliostato que possa ser utilizado para a produção de tijolos de barro cozido, utilizando a energia solar heliotérmica. Além disso, serão avaliados os benefícios econômicos e ambientais de se instalar este tipo de geração no sertão da Bahia. O protótipo desenvolvido incluirá engrenagens para amplificação do torque, potenciômetros de 10 voltas para controle preciso dos motores DC, um sistema de controle baseado em Arduino e uma central de controle com informações de data, hora, latitude e dia do ano.

-----------------------------------------------------

Esta etapa deverá estar subdividido em duas partes.

A primeira, chamada objetivo geral, é a definição do que será feito: explicar, entender, analisar, realizar uma coleta de dados, interpretar informações.

Já, na segunda, os objetivos específicos, estão diretamente relacionados com a justificativa, são aqueles que buscam demonstrar exatamente o que se pretende com o assunto.

Assim, deve-se sempre utilizar verbos no infinitivo para iniciar os objetivos:

Exploratórios (conhecer, identificar, levantar, descobrir)
Descritivos (caracterizar, descrever, traçar, determinar)
Explicativos (analisar, avaliar, verificar, explicar)
Este é o único capítulo de todo o Projeto que deve aparecer na forma de tópicos, já que os demais estarão em texto cursivo e problematizado.

Por isso ele é geralmente curto porque pode desvirtuar a pesquisa para sem que alcance os objetivos propostos.

Defina por exemplo se pretende dar uma solução definitiva ou se deseja apenas esclarecer melhor a questão por trás do que está pesquisando.


