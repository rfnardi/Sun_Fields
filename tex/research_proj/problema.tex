O problema de pesquisa terá formato de uma pergunta que deverá ser respondida no decorrer da pesquisa. Em geral, essa pergunta pode ser algo como “qual parcela da população mora com os pais?”.

Ou, “qual o processo mais viável economicamente para fazer um determinado produto?”.

Então, formule também uma hipótese de encontro ao problema de pesquisa: ela é a explicação de onde você acredita que a resposta para sua pergunta pode estar e onde você pretende dedicar seu tempo procurando.

Isso ajuda a definir o cronograma que será apresentado e ajuda a definir a importância do projeto de pesquisa.

Com a questão a ser respondida e a forma de encontrar a resposta definidos, o próximo passo é definir os objetivos do seu trabalho.
