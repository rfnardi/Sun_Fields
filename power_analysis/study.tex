\documentclass[12pt,a4paper]{article}

\usepackage[a4paper]{geometry}
\geometry{verbose,tmargin=1.5cm,bmargin=2cm,lmargin=2cm,rmargin=2cm}
\usepackage{amsmath,amsfonts,amssymb,amsthm,dsfont,mathtools}
\usepackage{cancel}
\usepackage{setspace}
\usepackage{booktabs}
\usepackage{hyperref}
\usepackage{graphicx}
\usepackage{bookmark}
\usepackage{verbatim}
\onehalfspacing
\usepackage[brazilian]{babel}

\usepackage{tikz, pgfplots}
\usepackage{tkz-fct, tkz-base, tkz-euclide} %tkz-fct chama gnuplot --> precisa compilar com a opção -shell-escape depois de pdflatex
\pgfplotsset{width=10cm,compat=1.17}
\usetikzlibrary{calc,angles,quotes,intersections}
\usetikzlibrary{decorations.pathmorphing}


\begin{document}
	
\begin{figure}
	% \centering
	\begin{tikzpicture}
		\tkzInit[xmin=0,xmax=7,ymin=-2,ymax=7]
		\tkzDefPoint(0,0){extr_left}
		\tkzDefPoint(5,0){extr_right}
		\tkzDrawSegment(extr_right,extr_left)
		\tkzDefPoint(0,0){Origem}
		\tkzDefPoint(0,4){Origem_acima}
		\tkzDrawSegment[->,>=stealth](Origem,Origem_acima)

		\tkzDefPoint(30:3){raio1_1}
		\tkzDefPoint(30:5){raio1_2}
		\tkzDrawSegment[->,>=stealth](raio1_2,raio1_1)
		\tkzDrawSegment[dashed,thin](Origem,raio1_1)
		\tkzDefPoint(30:0.5){angulo_apoio}
		\tkzLabelPoint[above](angulo_apoio){$\theta$}
		\tkzDrawArc(Origem,angulo_apoio)(Origem_acima)
		\tkzDefPoint(80:3){rotulo}
		\tkzLabelPoint(rotulo){\scriptsize Potência do raio incidente: $I_0$}
	\end{tikzpicture}
\end{figure}



\end{document}
