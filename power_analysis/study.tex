\documentclass[12pt,a4paper]{article}

\usepackage[a4paper]{geometry}
\geometry{verbose,tmargin=1.5cm,bmargin=2cm,lmargin=2cm,rmargin=2cm}
\usepackage{amsmath,amsfonts,amssymb,amsthm,dsfont,mathtools}
\usepackage{cancel}
\usepackage{setspace}
\usepackage{booktabs}
\usepackage{hyperref}
\usepackage{graphicx}
\usepackage{bookmark}
\usepackage{verbatim}
\onehalfspacing
\usepackage[brazilian]{babel}

\usepackage{tikz, pgfplots}
\usepackage{tkz-fct, tkz-base, tkz-euclide} %tkz-fct chama gnuplot --> precisa compilar com a opção -shell-escape depois de pdflatex
\pgfplotsset{width=10cm,compat=1.17}
\usetikzlibrary{calc,angles,quotes,intersections}
\usetikzlibrary{decorations.pathmorphing}


\begin{document}
	
	{\bf Problema de se determinar a potência do raio refletido a partir da
	densidade de potência do raio incidente como dependência do ângulo de
	incidência.}

	\bigskip

	A potência $P$ é a integral de superfície do fluxo de potência $\vec I_0$
	ao longo da superfície que recebe este fluxo. Assim,
	\begin{equation}
		P = \int \vec I_0 \cdot d\vec a .
	\end{equation}
	onde $ d\vec a$ é o elemento de área da superfície sobre a qual o fluxo do
	radiação incide. Considerando uma superfície plana, que faz um ângulo $\theta$ 
	com o raio de radiação incidente, temos que a potência dissipada na superfície é dada por
	\begin{equation}
		P = I_{0}A\cos(\theta)\ . 
	\end{equation}


\begin{figure}[h!]
	\centering
	\begin{tikzpicture}
		\tkzInit[xmin=0,xmax=7,ymin=-2,ymax=7]
		\tkzDefPoint(-1,0){extr_left}
		\tkzDefPoint(1,0){extr_right}
		\foreach \i in {1,2,...,12}{
			\tkzDefPoint(0.2*\i -1.4,0){I\i}
			\tkzDefPoint(0.2*\i-1.2,-0.2){II\i}
			\tkzDrawSegment(I\i,II\i)

		}
		\tkzDefPoint(0,0){Origem}
		\tkzDefPoint(0,4){Origem_acima}
		\tkzDrawSegment[->,>=stealth](Origem,Origem_acima)

		\tkzDefPoint(30:3){raio1_1}
		\tkzDefPoint(30:5){raio1_2}

		\tkzLabelPoint[above right](raio1_2){$I_0$}
		\tkzDrawSegment[->,>=stealth](raio1_2,raio1_1)
		\tkzDrawSegment[dashed,thin](Origem,raio1_1)
		\tkzDefPoint(30:0.3){angulo_apoio}
		\tkzLabelPoint[above](angulo_apoio){\scriptsize $\theta$}
		\tkzDrawArc(Origem,angulo_apoio)(Origem_acima)
		
		\tkzDefPointWith[orthogonal,K=0.2](Origem,raio1_1) \tkzGetPoint{base_left}
		\tkzDefPointWith[orthogonal,K=-0.2](Origem,raio1_1) \tkzGetPoint{base_right}
		\tkzDefMidPoint(Origem,base_right) \tkzGetPoint{base_middle_right}
		% \tkzDrawArc(Origem,base_middle_right)(extr_right)
		% \tkzLabelPoint[right](base_middle_right){\scriptsize $\theta$}
		\tkzGetPointCoord(base_right){B}
		\tkzDefPoint(\Bx,0){point_B}
		% \tkzDrawSegment[thin, dashed](base_right,point_B)

		\tkzGetPointCoord(base_left){BB}
		\tkzDefPoint(\BBx,0){point_BB}
		% \tkzDrawSegment[thin, dashed](base_left,point_BB)

		% \tkzDrawSegment(base_left,base_right)
		\tkzDefPointWith[orthogonal,K=-0.2](raio1_1,Origem) \tkzGetPoint{upper_left}
		\tkzDefPointWith[orthogonal,K=0.2](raio1_1,Origem) \tkzGetPoint{upper_right}
		\tkzDefLine[parallel=through upper_right](raio1_1,raio1_2) \tkzGetPoint{top_right}
		\tkzDefLine[parallel=through upper_left](raio1_1,raio1_2) \tkzGetPoint{top_left}
		\tkzDrawSegment[->,>=stealth](top_right,upper_right)
		\tkzDrawSegment[->,>=stealth](top_left,upper_left)

		\tkzDefMidPoint(raio1_2,top_right) \tkzGetPoint{mid_right}
		\tkzDefMidPoint(raio1_2,top_left) \tkzGetPoint{mid_left}
		\tkzDefMidPoint(raio1_1,upper_right) \tkzGetPoint{bottom_mid_right}
		\tkzDefMidPoint(raio1_1,upper_left) \tkzGetPoint{bottom_mid_left}
		\tkzDrawSegment[->,>=stealth](mid_right,bottom_mid_right)
		\tkzDrawSegment[->,>=stealth](mid_left,bottom_mid_left)

		\tkzInterLL(Origem,extr_right)(top_right,upper_right) \tkzGetPoint{crossing_right}
		\tkzInterLL(Origem,extr_left)(top_left,upper_left) \tkzGetPoint{crossing_left}
		\tkzDrawSegment[very thin, dotted](upper_left,crossing_left)
		\tkzDrawSegment[very thin, dotted](upper_right,crossing_right)

		\tkzDrawSegment(crossing_left,crossing_right)

	\end{tikzpicture}
	\caption{Incidência de um raio de luz com densidade perpendicular de potência $I_0$.}
\end{figure}



\end{document}
