\documentclass[12pt,a4paper]{article}
\usepackage[a4paper]{geometry}
\geometry{verbose,tmargin=1.5cm,bmargin=2cm,lmargin=2cm,rmargin=2cm}
\usepackage{amsmath,amsfonts,amssymb,amsthm,dsfont,mathtools}
\usepackage{cancel}
\usepackage{setspace}
\usepackage{booktabs}
\usepackage{hyperref}
\usepackage{graphicx}
\usepackage{bookmark}
\usepackage{verbatim}
\onehalfspacing
\usepackage[brazilian]{babel}
\usepackage{tikz, pgfplots}
\usepackage{tkz-fct, tkz-base, tkz-euclide} %tkz-fct chama gnuplot --> precisa compilar com a opção -shell-escape depois de pdflatex
\usepackage{animate}
\pgfplotsset{width=10cm,compat=1.17}
\usetikzlibrary{calc,angles,quotes,intersections}
\usetikzlibrary{decorations.pathmorphing}
\DeclareMathOperator{\sen}{sen}

\begin{document}
	{\bf Problema:} determinar como a potência refletida por um espelho em
	direção ao ponto focal depende da localização do espelho.

	\begin{figure}[htpb]
		\centering
			\begin{animateinline}[poster=first,controls={play,stop,step}]{4}
				\multiframe{35}{ri=0+4}
				{
					\begin{tikzpicture}[scale=1, transform shape]
						\tkzInit[xmin=-7,xmax=7,ymax=7]
						\tkzDrawX[noticks]
						\tkzDrawY[noticks]
						\tkzDefPoint(0,0){Origem}
						\tkzDefPoint(0.5,0.5){A}
						\foreach \i in {0,1,2,3}{
							\pgfmathparse{0+\i*90}
							\tkzDefPointBy[rotation=center Origem angle \pgfmathresult](A) \tkzGetPoint{A\i}
						}
						\tkzDrawPolygon[fill=red!80](A0, A1, A2, A3)

						\def\MirrorPos{4}
						\tkzDefPoint(\MirrorPos,\MirrorPos){P}
						\tkzDefPoint(\MirrorPos + 0.5,\MirrorPos + 0.5){B}
						\foreach \j in {0,1,2,3}{
							\pgfmathparse{0+\j*90}
							\tkzDefPointBy[rotation=center P angle \pgfmathresult](B) \tkzGetPoint{B\j}
						}
						\tkzDrawPolygon[fill=blue!30](B0, B1, B2, B3)
						\tkzDrawSegment[->,>=stealth](Origem,P)
						\tkzDefMidPoint(Origem,P) \tkzGetPoint{r}
						\tkzLabelPoint[above left](r){$\hat{r}$}

						\def\radiusSunPos{6}
						\tkzDefPoint(160 - \ri:\radiusSunPos){S}
						\tkzGetPointCoord(S){SCoord}
						\tkzDefPoint(\SCoordx + 0.3,\SCoordy + 0.3){C}
						\foreach \k in {0,1,2,3,4}{
							\pgfmathparse{0+\k*72}
							\tkzDefPointBy[rotation=center S angle \pgfmathresult](C) \tkzGetPoint{C\k}
						}
						\tkzDrawPolygon[fill=yellow!80](C0, C2, C4, C1, C3)
						\tkzDrawSegment[->,>=stealth](P,S)
						\tkzDefMidPoint(S,P) \tkzGetPoint{sun_to_mirror}
						\tkzLabelPoint[above left](sun_to_mirror){$\hat{s}$}
					\end{tikzpicture}
				}
			\end{animateinline}
		\caption{Reflexão em um espelho da malha: vista de cima}%
		\label{fig:um_espelho_da_malha}
	\end{figure}
	
	Para determinar a inclinação do espelho usamos o fato de que a normal do
	espelho $\vec n$ faz ângulos idênticos com a direção dos raios solares
	(dada pelo vetor $\hat{s}$) e com o vetor $\hat{r}$ que dá a localização do
	espelho em relação ao ponto focal aliado ao fato de que os três vetores
	mencionados pertencem ao mesmo plano (numa reflexão, os vetores raio
	incidente, raio refletido e normal da superfície são coplanares).

	\begin{figure}[htpb]
		\centering
		\begin{animateinline}[poster=first,controls={step,stop,step}]{8}
			\begin{tikzpicture}[scale=1, transform shape]
				\tkzInit[xmin=-4,xmax=4,ymax=4.5]
				\tkzDrawX[noticks]
				\tkzDrawY[noticks]
				\tkzDefPoint(0,0){Esp}
				\tkzDefPoint(0,4){Esp_above}
				\tkzDefPoint(2,2){s}
				\tkzDefPoint(1,1){right_alpha_above}
				\tkzDefPoint(1.41,0){right_alpha}
				\tkzDefPoint(18:1.41){right_alpha_label}
				\tkzLabelPoint[above right](right_alpha_label){$\alpha$}
				\tkzDrawArc[thick](Esp,right_alpha)(right_alpha_above)
				\tkzDrawArc[thick](Esp,right_alpha_above)(Esp_above)
				\tkzDefPoint(60:1.41){theta}
				\tkzLabelPoint[above](theta){$\theta$}
				\tkzDefPoint(-1,1){left_alpha_above}
				\tkzDefPoint(-1.41,0){left_alpha}
				\tkzDefPoint(162:1.41){left_alpha_label}
				\tkzLabelPoint[above left](left_alpha_label){$\alpha$}
				\tkzDrawArc[thick](Esp,left_alpha_above)(left_alpha)
				\tkzDefPoint(-2,2){r}
				\tkzDrawSegment[->,> = stealth, thick](r,Esp)
				\tkzDrawSegment[->,> = stealth, thick](Esp,s)
				\tkzDrawSegment[->,> = stealth, thick](Esp, Esp_above)
				\tkzLabelPoint[left](r){$\hat{r}$}
				\tkzLabelPoint[right](s){$\hat{s}$}
				\tkzLabelPoint[above right](Esp_above){$\vec n$}
			\end{tikzpicture}

			\newframe
			\begin{tikzpicture}[scale=1, transform shape]
				\tkzInit[xmin=-4,xmax=4,ymax=4.5]
				\tkzDrawX[noticks]
				\tkzDrawY[noticks]
				\tkzDefPoint(0,0){Esp}
				\tkzDefPoint(2,2){s}
				\tkzDefPoint(-2,2){r}
				\tkzDrawSegment[->,> = stealth, thick](r,Esp)
				\tkzDrawSegment[->,> = stealth, thick](Esp,s)
				\tkzDrawSegment[->,> = stealth, thick, red](r,s)
				\tkzDefMidPoint(r,s) \tkzGetPoint{midpoint}
				\tkzLabelPoint[above right, red](midpoint){$\hat{s} + \hat{r}$}
				\tkzDefMidPoint(Esp,s) \tkzGetPoint{s_label}
				\tkzDefMidPoint(Esp,r) \tkzGetPoint{r_label}
				\tkzLabelPoint[below right](s_label){$\hat{s}$}
				\tkzLabelPoint[below left](r_label){$\hat{r}$}
			\end{tikzpicture}

			\newframe
			\begin{tikzpicture}[scale=1, transform shape]
				\tkzInit[xmin=-4,xmax=4,ymax=4.5]
				\tkzDrawX[noticks]
				\tkzDrawY[noticks]
				\tkzDefPoint(0,0){Esp}
				\tkzDefPoint(2,2){s}
				\tkzDefPoint(-2,2){r}
				\tkzDrawSegment[->,> = stealth, thick](Esp,r)
				\tkzDrawSegment[->,> = stealth, thick](Esp,s)
				\tkzDefPoint(0,4){top} 
				\tkzDrawSegment[->,> = stealth, thick, red](r,top)
				\tkzDrawSegment[->,> = stealth, thick, red](s,top)
				\tkzDefMidPoint(Esp,s) \tkzGetPoint{s_label}
				\tkzDefMidPoint(Esp,r) \tkzGetPoint{r_label}
				\tkzDefMidPoint(s,top) \tkzGetPoint{s_label_top}
				\tkzDefMidPoint(r,top) \tkzGetPoint{r_label_top}
				\tkzDefMidPoint(Esp,top) \tkzGetPoint{sum_label}
				\tkzLabelPoint[right, red](sum_label){$\vec n$}
				\tkzLabelPoint[below right](s_label){$\hat{s}$}
				\tkzLabelPoint[below left](r_label){$-\hat{r}$}
				\tkzLabelPoint[above right](s_label_top){$-\hat{r}$}
				\tkzLabelPoint[above left](r_label_top){$\hat{s}$}
				\tkzDrawSegment[->,> = stealth, thick, red](Esp,top)
			\end{tikzpicture}

		\end{animateinline}
		\caption{Reflexão num espelho: vista de perfil no espelho}%
		\label{fig:reflexao}
	\end{figure}

	Assim, chegamos à expressão do vetor normal à superfície de tamanho unitário:
	\begin{eqnarray}
		\hat{n} &=& \frac{\vec n}{|\vec n|}\\
				&=& \frac{\hat{s} - \hat{r}}{|\hat{s} - \hat{r}|}\\
				&=& \frac{\hat{s} - \hat{r}}{\sqrt{s^2 -2\hat{s}\cdot\hat{r} + r^2}}\\
				&=& \frac{\hat{s} - \hat{r}}{\sqrt{1 -2\hat{s}\cdot\hat{r} + 1}}\\
				&=& \frac{\hat{s} - \hat{r}}{\sqrt{2}\sqrt{1 -\hat{s}\cdot\hat{r}}}\\
				&=& \frac{\hat{s} - \hat{r}}{\sqrt{2}\sqrt{1 -\cos(2\alpha)}}. \label{eq:n_quase_final}
	\end{eqnarray}

	Por fim, usamos as identidades trigonométricas 
	\begin{equation}
		1-\cos(2\alpha) = 2\sen^2(\alpha)
	\end{equation}
	e 
	\begin{equation}
		\sen(\alpha) = \cos\left(\frac{\pi}{2}-\alpha\right) = \cos(\theta)
	\end{equation}
	que levam (\ref{eq:n_quase_final}) à seguinte forma:
	\begin{equation}
		\boxed{\hat{n} = \frac{\hat{s} - \hat{r}}{2\cos(\theta)}\ .}
	\end{equation}
	
\end{document}
